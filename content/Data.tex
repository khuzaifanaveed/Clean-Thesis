\chapter{Data}
\label{chap:data}


The foundation of any standard siren cosmological analysis lies in the quality, completeness, and coverage of the data. In this thesis, we rely on two primary data sources: \ac{GW} observations from the \acf{LVK} detectors, and galaxy catalogs providing \ac{EM} redshift information. These datasets are used in tandem to statistically contruct a redshift prior for each \ac{GW} event and ultimately to constrain the Hubble constant $H_0$.

\section{\ac{GW} Event Data}

\subsection{The GWTC-3 Catalog}

The \ac{GW} data used in this study are obtained from the \textit{\acf{GWTC}}~\citep{abbott201gwtc1, abbott2021gwtc2, abbott2023gwtc3, abbott2024gwtc21}. This catalog comprises all \ac{CBC} events detected during the three observing runs (O1, O2, and O3) of the Advanced LIGO and Virgo detectors. \ac{GWTC} includes over 93 confident detections, primarily \acf{BBH} mergers, along with a few \acf{NSBH} and \acf{BNS} systems.

For each event, the catalog provides posterior samples for key source parameters inferred through Bayesian parameter estimation. These include the detector-frame component masses $(m_{1}^{\text{det}}, m_{2}^{\text{det}})$, the luminosity distance $d_L$, sky location (right ascension $\alpha$ and declination $\delta$), orbital inclination angle $\iota$, chirp mass $\mathcal{M}$, and various spin parameters (e.g., $\chi_{\text{eff}}$, $\chi_1$, $\chi_2$). These samples form the basis for constructing the luminosity distance posterior and are essential inputs to the standard siren inference pipeline.

%\subsection{Evant Parameter Estimation?}
%Is it needed to mention the parameter estimation? I think not.
%
\subsection{Event Selection Criteria}

To focus our analysis on dark siren cosmology, we restrict our sample to events that do not have any confirmed electromagnetic counterpart (i.e., no associated kilonova or GRB), have a network \ac{SNR} exceeding 11, have somewhat good sky localization, and are accompanied by publicly available posterior samples for distance and sky position.

These criteria are designed to ensure both the statistical robustness of the inference and compatibility with existing galaxy catalogs. After applying these cuts, a subset of 47 events, with 42 \acp{BBH}, 3
\acp{NSBH}, and 2 \acp{BNS}, is retained for cosmological analysis.

\subsection{Distance Posteriors and Sky Localization}

Each event is characterized by a posterior distribution over the three-dimensional localization volume, typically encoded using the HEALPix format. The posterior provides the probability density $p(d_L, \alpha, \delta)$, where $d_L$ is the luminosity distance and $(\alpha, \delta)$ denote right ascension and declination.

These distance posteriors are crucial for constructing the \ac{LOS} redshift prior when cross-matched with galaxy catalogs. Events with poor localization or multimodal distance distributions contribute more uncertainty to the inferred $H_0$, emphasizing the importance of event quality.

\subsection{Assumptions About the Source Population}
\label{sec:source_population}

Although this thesis primarily focuses on statistical redshift modeling, it is important to acknowledge that the cosmological inference also depends on assumptions about the source population. For instance, \texttt{gwcosmo} allows one to specify priors on:
\begin{itemize}
  \item Merger rate evolution with redshift, typically modeled as \( R(z) \propto (1+z)^\kappa \)
  \vspace{-1em}
  \item Mass distributions of the binaries
\end{itemize}

For our analysis we assume that \acp{CBC} follow a Madau-Dickinson merger redshift evolution model \citep{madau2014cosmic}:
\begin{align}
    R(z) = R_0(1+z)^{\gamma}\frac{1+(1+z_p)^{-(\gamma+\kappa)}}{1+\left( \frac{1+z}{1+z_p}\right)^{\gamma + \kappa}}
\end{align}
Here $R_0$ is the local merger rate, $\kappa$ the high-z slope, $\gamma$ the low-z slop and $z_p$ the break-point. All the assumed priors, taken from \citet{abbott2023gwtc}, are given in Table \ref{tab:Madau}.

To model the source population, for \ac{BBH}, we use power-law with a Gaussian peak mass model, with powerlaw slope $\alpha$, the mean of the Gaussian $\sigma_g$, the width of the peak $\mu_g$, and the relative weight between power-law and the peak $\lambda_g$. We also assume the minimum and maximum masses for the black hole distribution. For neutron stars we assume that mass is uniformly distributed between $M_{\mathrm{min, NS}}$ and $M_{\mathrm{max, NS}}$. All the assumed priors, taken from \citet{abbott2023gwtc}, are given in Table \ref{tab:mass_dist}.

\begin{table}[h!]
    \small
    \centering
    \caption{Priors on merger rate shape parameters.}
    \label{tab:Madau}
    \begin{tabular}{c l}
        \hline
        \textbf{Parameter} & \textbf{Prior} \\
        \hline
         $R_0$ & $1/R_0$ (implicit) \\
         $\kappa$ & $2.86$ \\
         $\gamma$ & $4.59$ \\
         $z_p$ & $2.47$ \\
         \hline
    \end{tabular}
\end{table}

\begin{table}[h!]
    \small
    \centering
    \caption{Priors on the mass model parameters.}
    \label{tab:mass_dist}
    \begin{tabular}{c l}
        \hline
        \textbf{Parameter} & \textbf{Prior} \\
        \hline
         $\alpha$ & $3.78$ \\
         $\sigma_g$ & $3.88$ \\
         $\mu_g$ & $32.27$ \\
         $\lambda_g$ & $0.03$ \\
         $M_{min,\mathrm{BH}}$ & $4.98M_{\odot}$ \\
         $M_{max,\mathrm{BH}}$ & $112.5M_{\odot}$ \\
         $M_{min,\mathrm{NS}}$ & $1.0M_{\odot}$ \\
         $M_{max,\mathrm{NS}}$ & $3.0M_{\odot}$ \\
         \hline
    \end{tabular}
\end{table}

\newpage

\section{\ac{EM} Galaxy Catalogs}

\subsection{The \texttt{GLADE+} Galaxy Catalog}

For redshift information, we use the \texttt{GLADE+} galaxy catalog~\citep{dalya2022glade+}, an extended version of the original \texttt{GLADE} catalog~\citep{dalya2018glade}, designed for \ac{GW} follow-up. \texttt{GLADE+} is a composite catalog that combines several large surveys to achieve all-sky coverage:
\vspace{-1em}
\begin{itemize}
  \item 2MASS XSC and 2MPZ: infrared-based all-sky surveys~\citep{skrutskie2006two, bilicki2013two} 
  \vspace{-1em}
  \item WISExSCOS: photometric redshifts from the \ac{WISE} and the SuperCOSMOS survey~\citep{bilicki2016wise}
  \vspace{-1em}
  \item HyperLEDA: spectroscopic redshift survey~\citep{makarov2014hyperleda}
  \vspace{-1em}
  \item SDSS DR16Q: quasar catalog~\citep{lyke2020sloan}
\end{itemize}

As of the latest release, \texttt{GLADE+} includes over 22 million objects with available positions, redshifts (spectroscopic or photometric), and photometry in multiple bands, most critically the $K$-band~\citep{dalya2022glade+}.

\subsection{Catalog Completeness}
\label{sec:luminosity_function}

A significant limitation of \texttt{GLADE+} is its incompleteness beyond redshift $z \sim 0.3$. While the catalog is approximately complete for bright galaxies at low redshift, its coverage of fainter galaxies or more distant regions is limited. This incompleteness introduces a bias when performing statistical redshift inference, as missing galaxies contribute to the out-of-catalog part of the redshift prior.

To address this, the \texttt{gwcosmo} pipeline models the galaxy population, for the out-of-catalog region, using a truncated Schechter luminosity function~\citep{schechter1976analytic} in the $K$-band. The Schechter function describes the distribution of galaxy luminosities and is given by:
\begin{align}
    \phi(M) \propto 10^{0.4(\alpha + 1)(M - M^*)} \exp[-10^{0.4(M - M^*)}]
\end{align}
where \( M \) is the absolute magnitude in the $K$-band, $M^*$ is the characteristic magnitude, and $\alpha$ is the faint-end slope. For our analysis, we adopt the standard parameters: $\alpha = -1.09$, $M^*_{K} = -23.39 + 5\log h$, consistent with \citet{kochanek2001k}, and truncate at $M^*_{K,\mathrm{min}} = -27.00 + 5\log h$ and $M^*_{K,\mathrm{max}} = -19.00 + 5\log h$.

\subsection{Redshift Uncertainty Modeling}

Redshift measurements in \texttt{GLADE+} are heterogeneous. Spectroscopic redshifts are generally precise, but photometric redshifts can have uncertainties on the order of $\sigma_z \sim 0.01-0.05$. These uncertainties are modeled by convolving the redshift of each galaxy with a Gaussian of fixed width, an assumption used in constructing the redshift prior.

Additionally, to reduce systematic error, galaxies with unphysical redshifts or unreliable photometry are filtered out during preprocessing. The net result is a cleaned, sky-localized, and redshift-tagged galaxy distribution that can be used for \ac{LOS} prior construction.

Furthermore, $K$-corrections are applied in \texttt{gwcosmo} following \citet{kochanek2001k}, and the apparent magnitude thresholds $m_{\mathrm{thr}}$ are computed per HEALPix pixel as a median apparent magnitude in a given pixel, giving a $K$-band threshold of 13.5 on average.

\subsection{$K$-band Luminosity as Host Probability Proxy}

In dark siren analysis, we must assign each galaxy a probability of being the true host. Following common practice, this probability is assumed to scale with stellar mass, for which $K$-band luminosity serves as a good proxy~\citep{strazzullo2006near,sureshkumar2021galaxy}. The host probability  $p_i$ for each galaxy is given by:
\begin{align}
    p_i \propto L_{K, i}
\end{align}
This choice is motivated by the assumption that more massive galaxies are more likely to host \acp{CBC} events. While simplistic, it provides a reasonable baseline.