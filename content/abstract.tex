% !TEX root = ../thesis-example.tex
%
\pdfbookmark[0]{Abstract}{Abstract}
\chapter*{Abstract}
\label{sec:abstract}
\vspace*{-13mm}
Gravitational-wave (GW) observations have introduced a powerful, independent method to measure the Hubble constant, $H_0$, through the use of so-called \textit{standard sirens}. This thesis investigates the use of \textit{dark sirens}, GW events without electromagnetic counterparts, for cosmological inference. In particular, it explores whether selecting the brightest galaxies as redshift tracers can improve the redshift prior and enhance the precision of $H_0$ estimation. By focusing on the most luminous galaxies, we partially mitigate the effects of catalog incompleteness currently plaguing dark siren measurements, effectively extending the reach of the catalog for cosmological analysis. Using the \texttt{gwcosmo} inference pipeline and the \texttt{GLADE+} galaxy catalog, a series of brightness-ranked percentiles were tested on a subset of the GWTC-3 catalog. The results show that moderate pruning improves $H_0$ constraints, while aggressive cuts lead to information loss and potential bias. Simulated mock data challenges using the \texttt{BUZZARD} catalog support this approach and reveal a lower pruning threshold near 30\% of the brightest galaxies. This method enhances the utility of incomplete galaxy catalogs in dark siren cosmology and contributes to the broader effort to resolve the Hubble tension.\\
\\

{\usekomafont{chapter} Samenvatting}\label{sec:abstract-nl}\\

\vspace*{-2mm}
Zwaartekrachtgolven (GW) bieden een krachtig en onafhankelijk middel om de Hubbleconstante, $H_0$, te meten via de zogenaamde \textit{standaard-sirenes}. Deze scriptie onderzoekt het gebruik van \textit{donkere sirenes}, GW-waarnemingen zonder elektromagnetische tegenhangers, voor kosmologische inferentie. In het bijzonder wordt nagegaan of het selecteren van de helderste sterrenstelsels als roodverschuivingstracers de $z$-prior kan verbeteren en zo de precisie van de $H_0$-schatting kan verhogen. Door ons te richten op de lichtkrachtigste sterrenstelsels wordt de impact van de onvolledigheid van bestaande sterrenstelscatalogi, een bekende uitdaging bij analyses met donkere sirenes, gedeeltelijk gemitigeerd, waardoor de effectieve diepte van de catalogus toeneemt. Met behulp van de \texttt{gwcosmo}-pipeline en de \texttt{GLADE+}-catalogus werd een reeks helderheidspercentielen getest op een subset van de GWTC-3-catalogus. De resultaten tonen aan dat gematigde afkappingen leiden tot een scherpere $H_0$ schatting, terwijl agressieve afkappingen resulteren in informatiederving en mogelijke bias. Gesimuleerde Mock Data Challenges met de \texttt{BUZZARD}-catalogus ondersteunen deze aanpak en wijzen op een praktische ondergrens van ongeveer 30\% van de helderste sterrenstelsels. Deze methode vergroot de toepasbaarheid van onvolledige sterrenstelscatalogi in donkere sirene-kosmologie en levert een bijdrage aan de bredere inspanningen om de Hubble-spanning op te lossen.
